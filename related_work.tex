\section{Related Work}

Password Based Signatures (PBS) the most similar primitive to Digital Consents could be considered. In PBS the client (in possession of a password) generates a public/private key pair $pk, sk$ to be used in signing.  The client transfers their signing key $sk$ to an agent, then the client makes a request to sign a message $m$. The client and the agent jointly computes a signature $\sigma$ on message $m$ which would verify under $pk$. Note that the password should be protected from offline dictionary attacks to keep the entropy requirement of the password in memorable levels.

\cite{GT12} formalized the notion of Password Based Signatures and provided two constructions based on RSA and CL blind signatures the latter being secure under GGM. \cite{JKR13} improved on \cite{GT12} by providing a PBS scheme using the BLS signature scheme proven secure under the random oracle model. They further construct a strongly secure variant of PBS in which smaller entropy password can be used without losing security.

Earlier \cite{HWF05} introduced server-aided digital signatures (SADS) which utilizes a two Server mechanism TODO

The main difference of digital consents from password based signatures is the fact that a digital consent is independent of the key material being used to do the signing procedure. Hence, it is not directly bind to a specific cryptographic operation such as signing. The aim of digital consents is to give authorization for an attribute (input) to be used along with a pre defined key material. This usage can be defined at an application level and hence be different than signatures. Due to this difference, the security model is also different. PBS has "unframability" meaning the Agent can not sign without the client's consent (or request according to PBS terminology), this is achieved by making the signing procedure a joint operation between the client and the server. In this work, the Agent can sign without the client's consent (the signing is done independent of the consent procedure). However due to the public verifiability and unforgeability properties of digital consents, there is a mechanism to detect whether this has been done.

\serge{Say that the PBS setting with a client having only a password and a server is not secure against offline password recovery by a malicious server. The SADS protocol fixes this by using two servers, at least one being honest.}