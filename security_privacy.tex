\section{Security/privacy issues}

\begin{itemize}
	\item  for the signature application, is that the agent can browse the internet to find a certificate matching the registered signing key, to identify the user. I already mentioned and the natural solution is to distribute the agent. There is yet another problem: if a signature by the user is released for some attributes (a document hash), the agent may find this signature on the internet as well. To solve this problem would require to change the signature protocol and solve two opposite constraints:
\begin{itemize}
\item we don't want the agent to "recognize" a signature (hence a blind signature)
\item we don't want the signer to be able to hide to the agent that a signature was signed by him in the dispute resolution protocol and this to make the agent unable to find the consent in the log.
\end{itemize}


\item There is also some kind of Sybil attack: if the user registers the same signing key under two different pseudo (pseudo1 and pseudo2), maybe even to two different agents, and if user launches the protocol with pseudo1 to sign a document and later say he is pseudo2 and complain because he did not sign the document, there is indeed no way to find the consent under pseudo2. Somehow, the "contract" on some valuable usage must be unique.

\item For the signing application, we could make the signing key be generated by the agent at enrolment and the signing key never to leave the agent so that there is no way to register the same key twice. But the problem to get a certificate and hide all these to the agent becomes more tricky. And we may have a complicated case if a user wants to change his agent. We may also use a central directory of "contracts" to identify duplicates.

\item For the key recovery application, I wonder what is the use case, the liability, the Sybil-resistance, and what is the dispute resolution protocol.
\end{itemize}

\section{Next}

\begin{itemize}
	\item Plug the above holes (binding to $\hat{X}$, anonymity to agent with signatures, Sybil attack, key recovery)
	\item Find a pairing-free variant
	\item Formalize the security/privacy notions
	\item Extend to multiple IdM, to t-out-of-n IdM, to distributed agents
	\item Determine if the password is useful
	\item Extend to multifactor (virtual IdM for some factors, OTP, Yubikey, TEE, OOB, FIDO, ...)
	\item Think of diverting IdM which are not designed for that
\end{itemize}